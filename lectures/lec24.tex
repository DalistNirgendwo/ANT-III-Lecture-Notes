\lecture{02.02.2024}
\section{Roth's theorem}

\begin{defn*}[Approximation exponent] \index{Approximation exponent}
	Let \( \alpha \in \R \).
	We define the \emph{approximation exponent} \( \tau_\alpha \) of \( \alpha \) as the infimum of real numbers \( \tau>0 \), such that for all \( \delta > 0 \) the inequality
	\[ |\alpha - \xi| \leq H(\xi)^{-\tau-\delta} \]
	has only finitely many solutions \( \xi \in \Q \).
\end{defn*}

\begin{rem*}
	\begin{enumerate}[label=\alph*)]
		\item For \( \xi = \frac{p}{q} \in \Q \) with \( p,q \in \Z \), \( q \neq 0 \), and \( \gcd(p,q) = 1 \), we have
			\[ H(\xi) = \max(|p|, |q|) \,. \]
		\item For \( \alpha \in \Q \) we have \( \tau_\alpha = 1 \).
		\item If \( \alpha \in \R\setminus \Q \), then by Dirichlet's theorem we have \( \tau_\alpha \geq 2 \).
	\end{enumerate}
\end{rem*}

\begin{thm*}[Roth, 1955] \index[theorem]{Roth's theorem}
	Let \( \alpha \in \R \cap \QQbar \) and \( \kappa > 2 \).
	Then there exists a constant \( c(\alpha, \kappa) > 0 \), such that
	\[ |\xi - \alpha| \geq c(\alpha, \kappa) H(\xi)^{-\kappa} \]
	for all \( \xi \in \Q \), \( \xi \neq \alpha \).
\end{thm*}

\begin{rem*}
	\begin{enumerate}[label=\alph*)]
		\item For all \( \alpha \in \R \cap \QQbar \) with \( \alpha \not\in \Q \), we deduce that the approximation exponent \( \tau_\alpha = 2 \).
		\item The constant \( c(\alpha, \kappa) \) in Roth's theorem is ineffective, i.e. with the methods of the proof one cannot compute \( c(\alpha, \kappa) \) explicitly.
	\end{enumerate}
\end{rem*}

We will prove a weaker result:

\begin{thm}[Liouville, 1844]
	Let \( \alpha \in \R \cap \QQbar \) of degree \( d \geq 1 \).
	Then there exits an effectively computable constant \( c(\alpha) > 0 \), such that
	\[ |\xi - \alpha| \geq c(\alpha) H(\xi)^{-d} \]
	for all \( \xi \in \Q \), \( \xi \neq \alpha \).
\end{thm}

\begin{rem*}
	If \( f(X) = a_0 X^d + \dots + a_d \in \Z[X] \) is a primitive polynomial of \( \alpha \), then we define the \emph{Mahler measure} of \( \alpha \) as
	\[ M(\alpha) = |a_0| \prod_{j=1}^d \max (1, |a^{(j)}|) \]
	and Liouville's theorem can be formulated as
	\[ |\alpha - \xi| \geq 2^{1-d} M(\alpha) H(\xi)^{-d} \]
	for all \( \xi \in \Q \), \( \xi \neq \alpha \).
\end{rem*}

\begin{conj*}[Lehmer, 1930s]
	There is a constant \( c>0 \), such that for all \( \alpha \in \QQbar \setminus \{0\} \), which is not a root of unity, one has
	\[ M(\alpha) \geq 1 + c \,. \]
\end{conj*}


\subsection*{Roth's theorem and Thue equations}

\begin{defn*}[Square-free binary form]
	We call a binary form \( F(X,Y) \in \Z[X,Y] \) \emph{binary-free} if it is not divisible by a square \( (\alpha X + \beta Y)^2 \in \C[X,Y] \), \( \alpha, \beta \in \C \) not both zero.
\end{defn*}

\begin{thm}
	Let \( F(X,Y) \in \Z[X,Y] \) be a square-free form of degree \( d \geq 3 \) and \( \kappa > 2 \).
	Then there exists a constant \( c(F,\kappa) > 0 \), such that for every \( (p,q) \in \Z^2 \) with \( F(p,q) \neq 0 \), we have
	\[ |F(p,q)| \geq c(F, \kappa) \max(|p|, |q|)^{d-\kappa} \,. \]
\end{thm}

\begin{cor}
	Let \( F(X,Y) \in \Z[X,Y] \) be a binary form, such that \( F(X,1) \) has at least three distinct roots, and \( m \in \Z \setminus \{0\} \).
	Then the equation \( F(X,Y) = m \) has at most finitely many integer solutions.
\end{cor}


\section{Schmidt's subspace theorem}

\begin{thm}[Schmidt, 1972]\label{thm:5.13} \index[theorem]{Subspace theorem}
	Let \( L_i(X_1, \dotsc, X_n) = a_{i1} X_1 + \dots + a_{in} X_n \), \( 1 \leq i \leq n \), be \( n \) linearly independent integer forms with \( a_{ij} \in \QQbar \), \( 1 \leq i, j \leq n \), \( n \geq 2 \).
	Let \( \delta, C > 0 \).
	Then all integer solutions \( x \in \Z^n \) to the inequality
	\[ |L_1(x) \dotsm L_n(x)| \leq C \|x\|_\infty^{-\delta} \]
	are contained in a finite union of proper linear subspaces of \( \Q^n \).
\end{thm}

%\begin{rem*}
%	\Cref{thm:5.13} implies Roth's theorem as follows:
%	Let \( \alpha \in \QQbar \setminus \Q \).
%	Define the two linear forms \( L_1(X_1, X_2) = X_1 - \alpha X_2 \) and \( L_2(X_1, X_2) = X_2 \).
%	Assume that \( p,q \in \Z \) with \( q \neq 0 \), \( \gcd(p,q)=1 \) and such that
%	\[ \left| \frac{p}{q} - \alpha \right| < C_1 \max (|p|, |q|)^{-2-\delta} \]
%	for some constant \( C_1 > 0 \), \( \delta > 0 \).
%	We find that
%\end{rem*}

\begin{frage*}
	Under which conditions does the inequality
	\begin{equation}\label{eq:5.3}
		|L_1(x) \dotsm L_n(x)| \leq C \|x\|_\infty^{-\delta}
	\end{equation}
	in Schmidt's subspace theorem have only \emph{finitely many} solutions?
\end{frage*}

\begin{exmp*}
	Let \( L_1(X) = a_{11} X_1 + \dots a_{1n} X_n \) with \( a_{11}, \dotsc, a_{1n} \in \Z \) and \( x^{(0)} = (x_1^{(0)}, \dotsc, x_n^{(0)}) \in \Z^n \setminus \{0\} \) a solution to \( L_1(x^{(0)}) = 0 \).
	Then \eqref{eq:5.3} has infinitely many solutions given for example by \( \lambda x^{(0)} \), \( \lambda \in \Z \).
\end{exmp*}

\begin{frage*}
	How about restricting to the inequality
	\[ 0 < |L_1(x) \dotsm L_n(x)| \leq C \|x\|_\infty^{-\delta} \,? \]
\end{frage*}