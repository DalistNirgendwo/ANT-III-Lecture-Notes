\lecture{03.11.2023}

\subsection*{An application of the norm map}

Given a number field \( K \) with ring of integers \( \Ocal_K \), how can we find \( \Ocal_K^* \), i.e. the units in \( \Ocal_K \)?
\begin{itemize}
	\item If \( \alpha \in \Ocal_K^* \), \( \alpha^{-1} \in \Ocal_K \) and \( 1 = N_{K/\Q}(\alpha \alpha^{-1}) = N_{K/\Q}(\alpha) \cdot N_{K/\Q}(\alpha^{-1}) \).
		By \Cref{cor:10}, \( N_{K/\Q}(\alpha),\ N_{K/\Q}(\alpha^{-1}) \in \Z \implies N_{K/\Q}(\alpha) = \pm 1 \).
	\item If \( \alpha \in \Ocal_K \) with \( N_{K/\Q}(\alpha) = \pm 1 \), then \( \alpha \in \Ocal_K^* \).
\end{itemize}

\begin{exmp*}
	Let \( d \in \Z \), \( d \) squarefree.
	Then, for \( a, b \in \Q \), \( N_{\Q[\sqrt{d}]/\Q}(a + \sqrt{d}b) = (a+b\sqrt{d}) (a - b\sqrt{d}) = a^2 - db^2 \).
	For \( d \equiv 2,3 \bmod 4 \), we find that
	\[ \Ocal_{\Q[\sqrt{d}]} = \set{a+b\sqrt{d}}{a,b \in \Z,\ a^2-db^2 = \pm 1}. \]
\end{exmp*}


\subsection*{The trace as a bilinear form}

Let \( L/K \) be number fields.
Then \( \Tr_{L/K} \) induces a bilinear form
\begin{equation}\label{eq:trace_binilear}
	\Tr_{L/K}: L \times L \to K,\ (x,y) \mapsto \Tr_{L/K}(x \cdot y).
\end{equation}
Write \( L^* \) for the dual vector space of \( L \), i.e. the set of all \( K \)-linear vector space homomorphisms.

\begin{thmn}
	The bilinear form \eqref{eq:trace_binilear} induces an isomorphism of \( K \)-vector spaces
	\[ \psi: L \to L^*,\ x \to \Tr_{L/K}(x,\cdot). \]
\end{thmn}

\begin{cor}
	Let \( L/K \) be number fields and \( (v_1, \dotsc, v_n) \) a \( K \)-basis with \( n = [L:K] \).
	Then there exists a unique \( K \)-basis \( (w_1, \dotsc, w_n) \) of \( L \), such that \( \Tr_{L/K}(v_iw_j) = \delta_{ij},\ 1 \leq i,j, \leq n \).
\end{cor}


\section{Discriminant}

Let \( K/\Q \) be a number field of degree \( n = [K:\Q] \) and \( \sigma_1, \dotsc, \sigma_n: K \to \C \) its embeddings.

\begin{defn*}[Discriminant]\index{Discriminant}
	For \( \alpha_1, \dotsc, \alpha_n \in K \), we define the \emph{discriminant} as
	\[ \disc(\alpha_1, \dotsc, \alpha_n) = \det\big( (\sigma_i(\alpha_j))_{1 \leq i,j \leq n} \big)^2. \]
\end{defn*}

\begin{thmn}
	Let \( \alpha_1, \dotsc, \alpha_n \in K \).
	Then \( \alpha_1, \dotsc, \alpha_n \) are \( \Q \)-linearly independent if and only if \( \disc(\alpha_1, \dotsc, \alpha_n) \neq 0 \).
\end{thmn}

\begin{lem}
	Let \( \alpha_1, \dotsc, \alpha_n \in K \). Then
	\[ \disc(\alpha_1, \dotsc, \alpha_n) = \det \big( \Tr_{K/\Q} (\alpha_i \alpha_j) \big)_{1 \leq i,j \leq n}. \]
\end{lem}

\begin{cor}
	Let \( \alpha_1, \dotsc, \alpha_n \in K \).
	Then \( \disc(\alpha_1, \dotsc, \alpha_n) \in \Q \).
	If moreover \( \alpha_1, \dotsc, \alpha_n \in \Ocal_K \), then \( \disc(\alpha_1, \dotsc, \alpha_n) \in \Z \).
\end{cor}

\begin{thmn}
	Let \( \alpha \) be algebraic over \( \Q \) with \( \big[ \Q[\alpha] : \Q \big] = n \), and \( \alpha_1, \dotsc, \alpha_n \) the \( n \) different conjugates of \( \alpha \).
	Then
	\[ \disc \big( 1, \alpha, \dotsc, \alpha^{n-1} \big) = \prod_{1 \leq i,j \leq n} (a_i - a_j)^2. \]
	If moreover \( f(x) \) is the minimal polynomial of \( \alpha \) over \( \Q \), then
	\[ \disc \big(1, \alpha, \dotsc, \alpha^{n-1} \big) = (-1)^\frac{n(n-1)}{2} N_{\Q[\alpha]/\Q} \big( (f'(\alpha) \big). \]
\end{thmn}

\begin{frage*}
	Let \( K \) be a number field with ring of integers \( \Ocal_K \) and of degree \( n = [K:\Q] \).
	Then \( K \) is an \( n \)-dimensional \( \Q \)-vector space.
	Hpw can we describe the structure of the group \( (\Ocal_K, +) \)?
\end{frage*}

\begin{exmp*}
	For \( d \in \Z \) squarefree and \( K = \Q \big[\sqrt{d}\big] \), the ring of integers \( \Ocal_K \) is a free abelian group of rank \( 2 \), where a \( \Z \)-basis is given by \( (1,\ \omega) \), with
	\[ \omega = \begin{cases}
		\sqrt{d} &d \equiv 2,3 \bmod 4,\\
		\frac{1 + \sqrt{d}}{2} &d \equiv 1 \bmod 4.
	\end{cases} \]
\end{exmp*}

\begin{thmn}\label{thm:20}
	Let \( K/\Q \) be a number field of degree \( n = [K:\Q] \).
	Then \( \Ocal_K \) is a free abelian group of rank \( n \), i.e. there exists \( \alpha_1, \dotsc, \alpha_n \in \Ocal_K \), such that every \( \beta \in \Ocal_K \) can be uniquely written in the form
	\[ \beta = m_1 \alpha_1 + \dots + m_n \alpha_n \]
	with \( m_1, \dotsc, m_n \in \Z \).
\end{thmn}

\begin{rem*}
	In the notation of \Cref{thm:20}, we call \( (\alpha_1, \dotsc, \alpha_n) \) and integral basis of \( \Ocal_K \) (over \( \Z \)).
\end{rem*}