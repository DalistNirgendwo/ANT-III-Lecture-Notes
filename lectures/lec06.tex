\lecture{17.11.2023}

\begin{thmn}
	Let \( R \) be a Dedekind domain and \( 0 \neq I \subseteq R \) an ideal.
	Then there is an ideal \( 0 \neq J \subseteq R \), such that \( IJ \) is principal.
\end{thmn}

\begin{exmp*}
	Let \( R = \Z\big[\sqrt{-5}\big] \) and \( I = \big(2, 1+\sqrt{-5}\big) \).
	Then \( I \) is not principal, but \( \big(2, 1+\sqrt{-5}\big)\big(2, 1-\sqrt{-5}\big) = (2) \) is principal.
\end{exmp*}

\begin{obs*}
	Note that \( \alpha \in I \) implies that \( J \subset A = \frac{1}{\alpha}IJ \).
	Hence \( \gamma JI = \gamma\alpha \left( \frac{1}{\alpha} JI \right) = \alpha\gamma A \subseteq (\alpha) \).
	As \( \gamma J \subseteq \gamma A \subseteq R \), we find that \( \gamma J \subseteq J \).
\end{obs*}

\subsection*{The ideal class group}

\begin{defn*}[Equivalence of ideals]
	Let \( R \) be an integral domain. We say that two non-zero ideals \( I, J \) are equivalent if and only if there exist \( \alpha, \beta \in R\setminus \{0\} \) with \( \alpha I = \beta J \).
\end{defn*}

\begin{rem*}\index{Ideal classes}
	\begin{enumerate}
		\item This really is an equivalence relation.
			We call the equivalence classes under this relation \emph{ideal classes}.
		\item We can define a multiplication on the set of ideal classes by multiplication of representatives, \( [I] [J] = [IJ] \), with the neutral element \( [R] \).
		\item All principal ideals form one ideal class.
	\end{enumerate}
\end{rem*}

\begin{cor}
	Let \( R \) be a Dedekind domain.
	Then the ideal classes form a group under multiplication.
\end{cor}

\begin{defn*}[Ideal class group]\index{Ideal classes!Ideal class group}
	We call the group given by ideal classes under multiplication in the Dedekind domain \( R \) the \emph{ideal class group} of \( R \), denoted by \( Cl(R) \).
\end{defn*}

\begin{exmp*}
	\( \Z \) is a principal ideal domain, hence \( |Cl(\Z)| = 1 \).
\end{exmp*}

\begin{rem*}
	There are only finitely many imaginary quadratic fields \( K \) with \( |Cl(\Ocal_K)| =~1 \).
\end{rem*}

\begin{frage*}[Gauss]
	Do there exist as many real quadratic number fields \( K \) with \( |Cl(\Ocal_K)| = 1 \)?
\end{frage*}

\begin{cor}
	Let \( R \) be a Dedekind domain and \( A,B,C \) ideals with \( A \neq 0 \).
	\begin{enumerate}
		\item If \( AB = AC \) then \( B=C \).
		\item We have \( B \divides A \), i.e. \( A = BJ \) for some ideal \( J \), if and only if \( A \subseteq B \).
	\end{enumerate}
\end{cor}

\begin{thmn}[Unique prime ideal factorisation]
	Every ideal \( I \neq 0 \) in a Dedekind domain \( R \) can be written as a product \( I = P_1 \dotsm P_r \) with non-zero prime ideals \( P_1, \dotsc, P_r \) and this representation is unique up to ordering of \( P_1, \dotsc, P_r \).
\end{thmn}

\begin{exmp*}
	In \( \Z\big(\sqrt{-5}\big) \) we don't have unique factorisation into reducible elements, e.g. \( 2 \cdot 3 = \big(1+\sqrt{-5}\big) \big(1 - \sqrt{-5}\big) \), but in terms of ideals we have \( (2) = \big(2, 1+\sqrt{-5}\big)^2 = P_1^2 \), \( (3) = \big(3, 1 + \sqrt{-5}\big) \big(3, 1 - \sqrt{-5}\big) = P_2 \cdot P_3 \). 
	Note that \( P_1, P_2, P_3 \) are all prime ideals as \( \left| \Z\big[\sqrt{-5}\big]/P_i \right| \in \{2,3\}  \) for \( 1 \leq i \leq 3 \). In the ideal class group we find that
	\begin{align*}
		(2)\cdot (3) &= P_1^2 P_2 P_3\\
		&= P_1 P_2 P_1 P_3\\
		&= \big(1 + \sqrt{-5}\big) \big(1 - \sqrt{-5}\big).
	\end{align*}
\end{exmp*}

\begin{defn*}[Greatest common divisor, least common multiple]
	Let \( R \) be a Dedekind domain and \( I, J \neq 0 \) ideals with prime factorisation
	\[ I = \prod_{i=1}^r P_1^{a_i},\ J = \prod_{i=1}^r P_i^{b_i}, \]
	where \( P_1, \dotsc, P_r \) are distinct prime ideals and \( a_1, \dotsc, a_r, b_1, \dotsc, b_r \in \Z_{\geq 0} \).
	We define the \emph{greatest common divisor} \( \gcd(I,J) \) and \emph{least common multiple} \( \lcm(I,J) \) by
	\[ \gcd(I,J) = \prod_{i=1}^r P_i^{\min(a_i,b_i)},\quad \lcm(I,J) = \prod_{i=1}^r P_i^{\max(a_i,b_i)}. \]
\end{defn*}

\begin{exc*}
	Show that
	\[ \gcd(I,J) = I + J,\quad \lcm(I,J) = I \cap J. \]
\end{exc*}

\begin{frage*}
	Given the ring of integers \( \Ocal_K \) in a number field \( K \), we know that every ideal is finitely generated.
	Can we say something about the numbers of generators we need?
	E.g. in \( \Z[\sqrt{-5}] \), the prime ideal \( (2, 1+\sqrt{-5}) \) is not a principal idea, but generated by two elements.
\end{frage*}