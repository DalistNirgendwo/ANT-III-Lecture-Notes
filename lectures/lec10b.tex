\section{Finiteness of the ideal class group}

Let \( K \) be a number field with ring of integers \( \Ocal_K \).
We will keep this notation throughout this chapter.

\begin{rec*}
	We call two non-zero ideals \( I,J \subseteq \Ocal_K \) equivalent, if \( \existss \alpha,\beta \in \Ocal_K \setminus \{0\} \), such that \( \alpha I = \beta J \), and we write \( Cl(\Ocal_K) \) for the group of equivalence classes under multiplication.
\end{rec*}

\begin{frage*}
	Is \( Cl(\Ocal_K) \) finite?
\end{frage*}

\begin{thmn}\label{thm:3.1}
	For every number field \( K \) there is a constant \( C_K \), such that every non-zero ideal \( I \) contains an element \( \alpha \in I \setminus \{0\} \) with
	\[ \big| N_{K/\Q} (\alpha) \big| \leq C_K N(I). \]
\end{thmn}

\begin{cor}\label{thm:3.2}
	Let \( K \) and \( C_K \) be as in \Cref{thm:3.1}.
	Then every ideal class \( C \in Cl(\Ocal_K) \) contains an ideal \( I \) with \( N(I) \leq C_K \).
\end{cor}

\begin{cor}\label{thm:3.3}
	For every number field \( K \) we have \( |Cl(\Ocal_K)| < \infty \).
\end{cor}

\begin{exmp*}
	Let \( K = \Q[\sqrt{2}] \), i.e. \( \Ocal_K = \Z[\sqrt{2}] \).
	As in the proof of \Cref{thm:3.1}, we can take \( C_K = (1 + \sqrt{2})^2 \) (using the integral basis \( (1, \sqrt{2}) \)).
	Note that \( (1 + \sqrt{2})^2 < 6 \).
	We consider the prime ideals in \( \Z[\sqrt{2}] \), which lie above 2, 3, 5.
	Note that \( 2 \Z[\sqrt{2}] = (\sqrt{2})^2 \) and that \( (3),\, (5) \) are prime ideals (see \Cref{thm:2.22}, noting that \( X^2-2 \) remains irreducible modulo 3, 5).
	Hence \( \big| Cl(\Z[\sqrt{2}]) \big| = 1 \).
\end{exmp*}

\begin{rem*}
	In the example above and other examples, we would like to take \( C_K \) as small as possible.
\end{rem*}

Our next goal will be to find improvements for the value of \( C_K \) using results from the geometry of numbers.

\begin{idee*}
	Let \( K \) be a number field of degree \( n \), \( \sigma_1, \dotsc, \sigma_r: K \hookrightarrow \R \) its real embeddings and \( \tau_1, \bar{\tau}_1, \tau_2, \bar{\tau}_2, \dotsc, \tau_s, \bar{\tau}_s: K \hookrightarrow \C \) its different complex embedings, where we sort them into pairs \( \tau_i, \bar{\tau}_i \), which differ by complex conjugations.
	Then \( n = r + 2s \) and we can define an injective map 
	\[ \varphi: K \to \R^n,\quad \alpha \mapsto \big( \sigma_1(\alpha), \dotsc, \sigma_r(\alpha), \Re \tau_1(\alpha), \Im \tau_1(\alpha), \dotsc, \Re \tau_s(\alpha), \Im \tau_s(\alpha) \big). \]
	Let \( (\alpha_1, \dotsc, \alpha_n) \) be an integral basis of \( \Ocal_K \).
	Then we can view \( \varphi(\Ocal_K) = \Z \varphi(\alpha_1) + \dots + \Z \varphi(\alpha_n) \subseteq \R^n \) as an additive group.
	Also, if \( I \subseteq \Ocal_K \) is a non-zero ideal, then \( I \) is a free \( \Z \)-module of rank \( n \), say with basis \( (\beta_1, \dotsc, \beta_n) \).
	Then
	\[ \varphi(I) = \Z \varphi(\beta_1) + \dots + \Z \varphi(\beta_n) \subseteq \R^n \]
	and we can interpret \( \varphi(I) \) as a \emph{lattice}\index{Lattice} in \( \R^n \).
	In order to improve upon \( C_K \) in  \Cref{thm:3.1}, we would like to find a "small" non-zero element in this lattice.
\end{idee*}