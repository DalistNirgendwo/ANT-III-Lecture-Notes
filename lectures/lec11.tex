\lecture{05.12.2023}
\section{Geometry of numbers}

Motivation: Consider a lattice \( L \), e.g. \( \Z^n \subseteq \R^n \), and a "nice" subset \( C \subseteq \R^n \), e.g. a ball of radius \( r \).
When does \( C \) contain a point in \( L \setminus\{0\} \)?

\begin{defn*}[Lattice]\index{Lattice}
	Let \( v_1, \dotsc, v_n \in \R^n \) be linearly independent vectors (over \( \R \)).
	Then we call the group
	\[ L = \set{z_1 v_1 + \dots + z_n v_n}{z_1, \dotsc, z_n \in \Z} \subseteq \R^n \]
	a (full) \emph{lattice} in \( \R^n \) and \( v_1, \dotsc, v_n \) a basis of \( L \).
	We define the determinant \( d(L) \) of the lattice \( L \) as
	\[ d(L) = |\det(v_1, \dotsc, v_n)|. \]
\end{defn*}

\begin{rem*}
	As additive groups we have \( L \cong \Z^n \).
	If \( x \in L \) and \( v_1, \dotsc, v_n \) as above, then there is exactly one way to write \( x \) as \( \sum_{i=1}^{n} x_i v_i \) with \( x_1, \dotsc, x_n \in \Z \).
\end{rem*}

\begin{notat*}
	We write \( M_{n \times n}(\Z) \) for the set of \( n \times n \) matrices with coefficients in \( \Z \). and \( GL(n, \Z) = \set{A \in M_{n \times n}(\Z)}{\det(M) = \pm 1} \) for the group of invertible matrices in \( M_{n \times n}(\Z) \).
\end{notat*}

\begin{lem}
	Let \( L \subseteq \R^n \) be a lattice and \( \{v_1, \dotsc, v_n\},\ \{w_1, \dotsc, w_n\} \) bases of \( L \).
	Then there exists a matrix \( A \in GL(n, \Z) \), say \( A = (a_{i,j})_{1 \leq i,j \leq n} \), such that
	\[ w_i = \sum_{i=1}^{n} a_{i,j}v_j,\quad 1 \leq i \leq n. \]
	Moreover,
	\[ |\det (v_1, \dotsc, v_n)| = |\det(w_1, \dotsc, w_n)|. \]
\end{lem}

\begin{rem*}
	In particular, the determinant \( d(L) \) of a lattice \( L \subseteq \R^n \) is well-defined.
\end{rem*}

Next, we want to compare the relative "size" of two lattices \( M \subseteq L \subseteq \R^n \).
Let \( L = \set{\sum_{i=1}^{n} z_i v_i}{z_1, \dotsc, z_n \in \Z} \) and \( M = \set{\sum_{i=1}^{n} t_i w_i}{t_1, dotsc, t_n \in \Z} \) with \( M \subseteq L \).
Then \( w_i \in L\ \foralll 1 \leq i \leq n \) and hence there exists an \( a_{i,j} \in \Z \) with \( w_i = \sum_{j=1}^{n} a_{i,j} v_j\ \foralll 1 \leq i \leq n \).
Let \( A = (a_{i,j})_{1 \leq i,j \leq n} \in M_{n \times n}(\Z) \).

\begin{defn*}[Index of a sublattice]
	In the notation above, we define the \emph{index} \( [L:M] \) of \( M \) in \( L \) as
	\[ [L:M] = |\det(A)|. \]
\end{defn*}

\begin{rem*}
	\begin{enumerate}
		\item The index \( [L:M] \) does not depend on the choice of bases of \( L,\, M \).
			By \( w_i = \sum_{j=1}^{n} a_{i,j}v_j \), we have
			\[ \underbrace{|\det(w_1, \dotsc, w_n)|}_{d(M)} = |\det(A)| \underbrace{|\det(v_1, \dotsc, v_n)|}_{d(L)}, \]
			and hence \( [L:M] = \frac{d(M)}{d(L)} \).
		\item One can show that \( [L:M] = |L/M| \), where \( L/M \) is the quotient group.
	\end{enumerate}
\end{rem*}

\begin{exmp*}
	Let \( e_1, \dotsc, e_n \) be the unit vectors in \( \R^n \), i.e. \( e_i = (0, \dotsc, 0, 1, 0, \dotsc, 0) \).
	\begin{enumerate}
		\item \( \Z^n = \set{\sum_{i=1}^{n} e_i z_i}{z_1, \dotsc, z_n \in \Z} \) is a lattice with \( d(\Z^n) = 1 \).
			Let \( d_1, \dotsc, d_n \in \N \) and set \( w_i = d_i e_i \) for all \( 1 \leq i \leq n \).
			Then \( M = \set{\sum_{i=1}^{n} z_i w_i}{z_1, \dotsc, z_n \in \Z} \subseteq \Z^n \) is a sublattice with \( d(M) = |\det(d_1e_1, \dotsc, d_ne_n)| = d_1 \dotsm d_n \) and \( [Z^n:M] = d_1 \dotsm d_n \).
			Hence, as abelian groups, \( \Z^n/M \cong \Z/d_1 \Z \oplus \dots \oplus \Z/d_n\Z. \)
		\item \( L = \set{\frac{a_1}{2} e_1 + \dots + \frac{a_n}{2} e_n}{a_1, \dotsc, a_n \in \Z,\ a_1 \equiv \dots \equiv a_n \bmod 2} \) is a lattice in \( \R^n \) with basis \( e_1, \dotsc, e_{n-1}, \frac{e_1 + \dots + e_n}{2} \).
	\end{enumerate}
\end{exmp*}

\subsection*{Convex bodies}

\begin{defn*}[Convex set]
	We call a subset \( C \subseteq \R^n \) \emph{convex} if for all \( x, y \in C \) the line segment 
	\[ \set{tx + (1-t)y}{0 \leq t \leq 1} \]
	is contained in \( C \) as well.
\end{defn*}

\begin{defn*}[Central symmetric convex body]
	A subset \( C \subseteq \R^n \)´is called a \emph{central symmetric convex body} if it has the following properties:
	\begin{enumerate}[label=(\alph*)]
		\item \( C \) is compact (i.e. closed and bounded) and convex. (convex body)
		\item \( 0 \) is in the interior of \( C \). (central)
		\item If \( x \in C \), then \( -x \in C \). (symmetric)
	\end{enumerate}
\end{defn*}

\begin{exmp*}
	\begin{enumerate}
		\item Let \( C \subseteq \R^n \) be a central symmetric convex body and \( A: \R^n \to \R^n \) an invertible linear map.
			Then \( A(C) \) is a central symmetric convex body.
		\item The norm \( \|x\|_2 = \left( \sum_{i=1}^{n} |x_i|^2 \right)^\frac{1}{2} \) leads to the \( n \)-dimensional unit ball
			\[ B_n = \set{x \in \R^n}{\|x\|_2 \leq 1}. \]
			\( \|x\|_\infty = \max_{1 \leq i \leq n}|x_i| \) induces the \( n \)-dimensional unit cube
			\[ K_n = \set{x \in \R^n}{\max_{1 \leq i \leq n} |x_i| \leq 1}. \]
			\( \|x\|_1 = \sum_{i}^{n} |x_i| \) give the \( n \)-dimensional unit octahedron
			\[ O_n = \set{x \in \R^n}{\sum_{i=1}^{n}|x_i| \leq 1}. \]
	\end{enumerate}
\end{exmp*}

\begin{lem}
	Let \( \|\cdot\|: \R^n \to \R^n_{\geq 0} \) be a norm.
	Then \( B_{\|\cdot\|} = \set{x \in \R^n}{\| x \| \leq 1} \) is a central symmetric convex body.
\end{lem}

So far we have found that every norm on \( \R^n \) "produces" a central symmetric convex body in \( \R^n \).
Is there a one-to-one correspondence, i.e. are these all the different classes of central symmetric convex bodies?

\begin{rem*}
	Let \( C \subseteq \R^n \) be a central symmetric convex body.
	For \( \lambda \geq 0 \), set \( \lambda C = \set{\lambda x}{x \in C} \).
	If \( \lambda > 0 \), then \( \lambda C \) is again a central symmetric body.
	For \( x \in \R^n \), we define \( \|x\|_C = \min \set{\lambda \in \R_{\geq 0}}{x \lambda \in C} \).
\end{rem*}

\begin{lem}
	Using the same notation as above, the following statements hold:
	\begin{enumerate}
		\item \( \| \cdot \|_C \) is well-defined.
		\item \( \| \cdot \|_C \) defines a norm on \( \R^n \).
		\item \( \lambda C = \set{x \in \R^n}{\|x\|_C \leq \lambda} \) for \( \lambda > 0 \).
	\end{enumerate}
	In particular, we recover \( C \) via \( C = \set{x \in \R^n}{\|x\|_C \leq 1} \).
\end{lem}

\subsection*{Minkowski's first convex body theorem}

\begin{wrapfigure}{r}{7cm}
	\wrapincfig{11_lattice_lambdaC}{7cm}
\end{wrapfigure}
Let \( L \subseteq \R^n \) be a lattice and \( C \subseteq \R^n \) a central symmetric convex body.
When is \( C \cap L \neq \{0\} \), i.e. when does \( C \) contain more lattice points than just \( 0 \)?

\begin{thm}[Minkowski's first convex body theorem, 1896]
	With the same notation as above, let \( \vol(C) \geq 2^n d(L) \).
	Then \( C \cap L \neq \{0\} \), i.e. there exists a \( x \in L \setminus \{0\} \) with \( x \in C \).
\end{thm}